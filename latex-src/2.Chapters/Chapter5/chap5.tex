\chapter{نتیجه‌گیری و پیشنهاد کارهای آینده}
ماشین‌های حالت مایع تلاش بر ارائه‌ی مدلی برای شبیه‌سازی ریزدارات قشری مغز دارند که توانایی طبقه‌بندی سیگنال‌های محرک پیچیده‌ی فضایی‌زمانی را با دقت قابل قبول داشته باشد. علاوه بر عناوینی که به آن پرداخته شد، موارد بسیاری در این حوزه وجود دارد که بعضا با وجود آنکه تا حدی به آنها در ادبیات موضوع پرداخته شده است، اما هنوز برداشت صحیح و کاملی وجود ندارد:

\begin{itemize}
\item
از ایده‌هایی که می‌توان برای بهبود عملکرد ماشین حالت مایع استفاده کرد، استفاده از چند فیلتر مایع به طور موازی است که هر کدام نمایش مختلفی از محرک ورودی را نتیجه بدهند. همچنین می‌توان به طور موازی چند لایه‌ی قرائت تعریف کرد که هر کدام تنها از بخشی از حالت مایع سراسری را در بر گرفته و به عبارتی هر کدام صرفا به بخشی از فضای مسئله دسترسی داشته باشند. این حوزه شباهت بسیاری به ایده‌ی ترکیب خبره‌ها داشته و امکان بهره بردن از ایده‌های این حوزه برای بهبود عملکرد ماشین حالت مایع وجود دارد.

\item
الگوهای اتصال بسیار بیشتری وجود دارد که می‌توان عملکرد آنها را تحت چارچوب ماشین‌های حالت مایع بررسی کرد. یک مرجع مناسب، مدل‌های معرفی شده در \cite{kaiser2011tutorial} توسط \lr{Kaiser} است که به بررسی مدل‌های گوناگون، از جمله دنیای کوچک و مستقل از مقیاس\footnote{\lr{Scale-free}}، پرداخته است.

\item
معمولا یادگیری توسط سیناپس‌های داخل ماشین حالت مایع صورت نمی‌پذیرد؛ هر چند که مدل‌هایی ارائه شده است که از تکنیک \lr{STDP} به این منظور استفاده کرده‌اند اما عموما با توجه به پیچیدگی محاسباتی تحمیل شده، بهبود قابل‌توجهی حاصل نشده است. روش دیگری که می‌توان به کار گرفت پازش\footnote{\lr{Refinement}} شبکه در طول زمان با حذف و اضافه‌ی اتصالات نورونی است که کار قابل توجهی هنوز در این زمینه صورت نگرفته است.

\item
تا جایی که نویسنده مطلع است کاری در زمینه‌ی طبقه‌بندی سیگنال ویدئویی توسط ماشین‌های حالت مایع صورت نگرفته است. با این حال ماشین حالت مایع بدلیل اینکه از زیرساخت عصبی ضربه‌ای تشکیل شده است، توانایی خوبی در انتقال محرک بر اساس زمان وقوع آن داشته و با وجود اینکه در اینجا نیز هر تصویر توسط یک سیگنال فضایی‌مکانی نمایش داده شده است، LSM برای کاربردهای مرتبط با حرکت\footnote{\lr{Motion}} می‌تواند پتانسیل بسیاری داشته باشد.

\item
در اینجا بدلیل محدودیت‌های محاسباتی از مدل نورونی تجمیع و آتش نشتی استفاده شد. مدل‌های نورونی بسیار متنوعی با سطوح مختلف انطباق زیستی وجود دارد که می‌توان عملکرد آنها را بررسی و در نتیجه مدل را به واقعیت زیستی نزدیک‌تر ساخت. نویسنده در ابتدا قصد داشت تا سازوکار ماشین حالت مایع را بر اساس مدل‌های نورونی دقیق‌تری همانند مدل نورونی \lr{Izhikevich} پیاده‌سازی کند که با توجه به نتایج نامناسب از ادامه‌ی کار صرف نظر شد. در \cite{grzyb2009model} مطالعه‌ی مختصری روی مدل‌های نورونی مختلف به عنوان بلوک‌های سازنده‌ی \lr{LSM} صورت گرفته است. مشکل مشهود برای مدل نورونی \lr{Izhikevich} موج‌های دوره‌ای خود‌به‌خودی است که در فازهای مختلف اجرا از خود نشان می‌دهد و باعث کم اثر شدن محرک در حالت فیلتر مایع می‌گردد. با این حال پدیده‌ی \lr{Polychronization} \cite{izhikevich2006polychronization} انگیزش زیادی برای تحقیق در این حوزه ایجاد می‌کند.
\end{itemize}

