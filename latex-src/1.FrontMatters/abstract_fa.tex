\begin{abstract_fa}

توانایی خارق‌العاده‌ی مغز در پردازش حجم بالایی از اطلاعات به صورت موازی و ساختار و سازمانش آن را بدون شک به پیچیده‌ترین ارگان شناخته شده در گیتی تبدیل کرده است. با وجود شباهت‌های آناتومی و فیزیولوژی مغز انسان، و با در نظر گرفتن عملکرد پیمانه‌ای مغز، مدارات محلی قشرهای گوناگون مغز تا حد زیادی به صورت تصادفی تشکیل شده است. گواه تشکیل این مدارات تصادفی عدم امکان در بر گرفتن توصیف نجومی اتصالات مغز در ژنوم است که ما را به بررسی امکان‌پذیری مشارکت شبکه‌های تصادفی در پردازش اطلاعات و توانمندی آنها سوق می‌دهد. از این روی در این تحقیق به توانایی بازشناسی الگو با بهره‌گیری از محاسبات انباره‌ای پرداخته می‌شود. ماشین حالت مایع که نمود محاسبات انباره‌ای بر مبنای مکانیسم نورون‌های مغز است اخیرا توجه زیادی را به خود جلب کرده است تا حدی که هم اکنون برای برخی از کاربردهای مهندسی نیز پیشنهاد می‌شود. ماشین‌های حالت مایع محرک ورودی را به فضایی با ابعاد بالاتر نگاشت می‌دهند که به موجب آن تفکیک الگوهای زمانی فضایی را ساده‌تر می‌نمایند؛ همانند چیزی که به موجب کرنل‌ها در ماشین‌های بردار پشتیبانی روی می‌دهد، با این تفاوت که ماشین حالت مایع از نظر زیستی انطباق بیشتری دارد. در این تحقیق با بررسی مکانیسم‌های مختلف اتصالات سیناپسی به واکاوی تاثیر توپولوژی شبکه روی عملکرد ماشین حالت مایع می‌پردازیم. در اینجا ارائه‌ی روش جدیدی برای نمایش حالت فیلتر مایع که به امکان تمایز قابل قبول بین نمونه‌ها بیانجامد ارائه شده است و همچنین معیاری جدید برای سنجش توانایی تفکیک ذاتی ماشین حالت مایع عرضه می‌شود.

\end{abstract_fa}
\begin{keyword_fa}
محاسبات انباره‌ای\sep
ماشین حالت مایع\sep
شبکه‌ی عصبی ضربه‌ای\sep
بازشناسی الگوی بصری
\end{keyword_fa}